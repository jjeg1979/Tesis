\chapter{EVALUACIÓN DE LAS HIPÓTESIS DEL MODELO} \label{cap:EvaluacionHipotesisModelo}
\vspace{0.2cm}
\noindent\rule{\linewidth}{1.5pt}\\
\startcontents[chapters]
\printcontents[chapters]{}{1}{}
\vspace{0.2cm}
\noindent\rule{\linewidth}{1.3pt}\\
\newpage

\section{Generación de Aglomerados Fractales Tridimensionales}\label{sec:GeneracionAglomeradosFractales3D}
\subsection{Introducción}
\subsection{Motivación}
\section{Planteamiento como problema de optimización}\label{sec:PlanteamientoProblemaOptim}
\section{Solución al problema}\label{sec:SolucionProblema}
\subsection{Metodología}
\subsection{Condiciones de convergencia}
\subsection{Validez y rango de aplicación}
\section{Generación de imágenes bidimensionales}\label{sec:GeneracionImagenes2D}
\section{Método de Box--Counting}\label{sec:MetodoBoxCounting}
\subsection{Desarrollo}
\subsection{Aplicación a las condiciones de contorno del método}
\subsection{Validación con aglomerados generados aleatoriamente}
\section{Estudio de la Monodispersidad}\label{sec:EstudioMonodispersidad}
\section{Estudio de la Esfericidad}\label{sec:EstudioEsfericidad}
\section{Resumen}



\newpage
%\bibliographystyle{plain}
\bibliographystyle{eunsrt}
% \bibliographystyle{apalike}
% \bibliographystyle{aea}
\bibliography{../References/Tesis}
%\putbib{Tesis}