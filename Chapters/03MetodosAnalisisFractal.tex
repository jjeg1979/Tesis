\chapter{MÉTODOS DE ANÁLISIS FRACTAL}\label{cap:MetodosAnalisisFractal}
%\vspace{0.1cm}
\noindent\rule{\linewidth}{1.3pt}\\
\startcontents[chapters]
\printcontents[chapters]{}{1}{}
%\vspace{0.1cm}
\noindent\rule{\linewidth}{1.1pt}\\
%\minitoc
\newpage
\section{Geometría fractal}\label{sec:GeometriaFractal}
\section{Conceptos básicos}\label{sec:ConceptosBasicosDimension}
\subsection{Concepto de Dimensión}
\subsection{Dimensión fractal}
\section{Ejemplos de fractales}\label{sec:EjemplosFractales}
\section{Hipótesis del modelado fractal}\label{sec:HipotesisModeladoFractal}
\section{Métodos del análisis fractal}\label{sec:MetodosAnalisisFractal}
\subsection{Métodos de Box--Counting}
\subsection{Métodos Multifractales}
\subsection{Métodos de Sistemas Funciones Iteradas (IFS)}
\section{Aplicaciones prácticas del análisis fractal}
\subsection{Análisis fractal de partículas diésel}
\subsection{Ejemplos en las ramas de la Biología y de la Botánica}
\subsection{Ejemplos en las ramas de la Geología y de la Geofísica}


\newpage
\bibliographystyle{ealpha}	
\bibliography{Chapters/Bibliografia}