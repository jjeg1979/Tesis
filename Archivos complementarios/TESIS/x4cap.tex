\section{\ce{C_c}-derivados del \cosane\  con grupos silano.}
\subsection{S�ntesis.}\label{sintesis}
En este apartado se describe la s�ntesis y caracterizaci�n de unos nuevos derivados de cobaltacarborano, que contienen grupos silano enlazados exo-cl�ster a los carbonos de los ligandos \ce{[C2B9H11]^{2-}}, utilizando como productos de partida \Cscosane, Cs[\textbf{1}] y \Cscosanep, Cs[\textbf{2}]. El principal objetivo es la preparaci�n de derivados que contengan una funci�n \ce{Si-H} en el cl�ster.


\subsection{Caracterizaci�n.}
Los compuestos han sido caracterizados por IR, RMN de \ce{^1H}, \ce{^{11}B}, \ce{^{13}C} y \ce{^{29}Si}, en algunos casos COSY \ce{^{11}B}\{\ce{^1H}\}-\ce{^{11}B}\{\ce{^1H}\}-RMN, espectrometr�a de masas, analisis elemental y los compuestos \ce{[N(CH3)4]}[\textbf{3}], \ce{[N(CH3)4]}[\textbf{4}] y \ce{[N(CH3)4]}[\textbf{7}]  por difracci�n de rayos X.
\subsubsection{Espectroscop�a de Infrarrojo (IR)}
La frecuencia de vibraci�n de los enlaces \ce{C_c-H} aparece como absorciones finas y poco intensas, entre 3090 y 3034 cm$^{-1}$. Entre 2584 y 2554 cm$^{-1}$ aparece una absorci�n muy intensa correspondiente a $\nu$\ce{(B-H)}; la zona donde aparece esta absorci�n es la caracter�stica para los carboranos tipo \textit{nido}. Una se�al caracter�stica de los compuestos \ce{\textbf{[3]}^-}, \ce{\textbf{[5]}^-} y \ce{\textbf{[9]}^-} es la correspondiente a la frecuencia de vibraci�n del enlace \ce{Si-H}, que aparece entorno a 2160 cm$^{-1}$ y que nos confirma la presencia de la funci�n Si-H. Otra banda com�n a todos los compuestos es la correspondiente a $\nu$\ce{(Si-CH3)}, que aparece entre 1250 y 1257 cm$^{-1}$.
\subsubsection{\ce{^1H}-RMN}
Los espectros de \ce{^1H}-RMN se han realizado en acetona deuterada y los desplazamientos qu�micos est�n referidos a TMS. En la Tabla \ref{Tabla2} se recogen los desplazamientos qu�micos de prot�n. A efectos comparativos se han incluido los productos de partida \ce{\textbf{[1]}^-} y \ce{\textbf{[2]}^-} en la tabla.
\begin{table}[htbp]
\begin{center}
\begin{tabular}{c c c c c}
\hline
compuesto & \ce{Si-H} & \ce{C_c-H} & \ce{Si-CH3} & \ce{B-H_{terminal}}  \rule{0in}{3ex} \\  \hline\hline
\ce{\textbf{[1]}^-} & - & 3.94 & - & 3.37-1.57  \rule{0in}{3ex} \\  
\ce{\textbf{[3]}^-} & 4.31 & 3.85, 3.69 & 0.29 & 3.61-1.60  \rule{0in}{3ex} \\  
\ce{\textbf{[4]}^-} & - & 4.5 & 0.31 & 3.38-1.43  \rule{0in}{3ex} \\  
\ce{\textbf{[5]}^-} & 5.06 & 4.59 & 0.44 & 3.40-1.44  \rule{0in}{3ex} \\  
\ce{\textbf{[6]}^-} & - & 4.02, 3.83, 3.72 & 0.28 & 3.57-1.50  \rule{0in}{3ex} \\  
\ce{\textbf{[7]}^-} & - & 4.19, 3.77 & 0.33, 0.30 & 3.98-1.58  \rule{0in}{3ex} \\  \hline
\ce{\textbf{[2]}^-} & - & 3.58 & - & 3.76-1.49  \rule{0in}{3ex} \\  
\ce{\textbf{[8]}^-} & - & 3.48 & 0.39, 0.25 & 4.00-1.43  \rule{0in}{3ex} \\  
\ce{\textbf{[9]}^-} & 5.25, 4.97 & 3.61 & 0.48, 0.38 & 4.11-1.43  \rule{0in}{3ex} \\  
\ce{\textbf{[10]}^-} & - & 3.52, 3.45 & 0.28 & 3.99-1.51  \rule{0in}{3ex} \\  \hline
\end{tabular}
\end{center}
\caption{Desplazamientos qu�micos de los protones (ppm) en el espectro de \ce{^1H}-\{\ce{^{11}B}\}-RMN.}
\label{Tabla2}
\end{table}



