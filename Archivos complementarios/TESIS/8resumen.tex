\cdpchapter{Abstract}
This work has open new strategies in the synthesis of large molecules, such as dendrimers and metallodendrimers, and other nanostructured materials, in the boron chemistry field. 

The main aim of this work was the preparation of polyanionic boron-rich metallodendrimers containing cobaltabisdicarbollide derivaties at the periphery, with potential applications  in biomedicine. For this purpose a set of novel \ce{C_c}-mono- and \ce{C_c}-disubstituted cobaltabisdicarbollide derivatives with silyl functions, \ce{\textbf{[3]}^-}--\ce{\textbf{[10]}^-}, have been prepared by the reaction of lithium salts of \cosane, \ce{\textbf{[1]}^-}, and \cosanep, \ce{\textbf{[2]}^-},  with different chlorosilanes. DFT theoretical studies at the B3LYP/6-311G(d,p) level of theory were applied to optimise the geometries of these compounds and calculate their relative energies, showing a good concordance between theoretical and experimental results. The unexpected formation of a bridge -$\mu$\ce{- SiMe2 -} between both dicarbollide clusters, through the \ce{C_c} atoms, after the reaction of the monolithium salt of cobaltabisdicarbollide with \ce{HSiMe2Cl}, suggested an intramolecular reaction, in which the acidic \ce{C_c-H} proton reacts with the hydridic \ce{Si-H}, with subsequent loss of \ce{H2}. Some aspects of this reaction have been studied by using DFT and QTAIM calculations. 

From all the previous compounds, the anion \cosaneSiH, \ce{\textbf{[5]}^-}, was chosen as hydrosilylating agent for the preparation of different types of metallodendrimers. Thus, different generations of polyanionic metallacarborane-containing metallodendrimers were constructed via hydrosilylation of various generation of carbosilane and cyclic carbosiloxane dendrimers containing terminal vinyl functions with \ce{\textbf{[5]}^-}, to achieve the corresponding metallodendrimers with four and eight peripheral cobaltacarboranes. For metallodendrimers with high molecular weights, the UV-Vis spectroscopy was used for corroborating the full functionalization and consequently the unified character of dendrimers. The solubility of these dendrimers is very interesting from the point of view of potential applications, i.e. in medicine or BNCT. For that reason, some solubility studies have been carried out by using UV-Vis measurements in water/DMSO solutions of these metallodendrimers. 

Following the same strategy, poly(aryl-ether) type dendrimers with a fluorescente core and peripheral allyl functions have also been hydrosilylated using the anion \ce{\textbf{[5]}^-}, to obtain metallodendrimers with three, six and twelve cobaltacarborane moieties. It is important to emphasize that photoluminescent measured on these compounds, showed that after functionalization, the presence of metallacarboranes at the periphery causes a quenching of the fluorescence previously exhibited by the starting dendrimers. Nowadays, we have not the explication to this phenomenon that is still under study.

Other type of polyanionic poly-(alkyl aryl-ether) metallodendrimers have also been prepared by using the ring opening reaction of the 8--dioxanate in \cdiox, by the nucleophilic attack to the oxygen with the alcoholate functions obtained by deprotonation of the alcohol groups (\ce{- OH}) located at the starting dendrimers periphery.

Carborane-containing siloxane and octasilsesquioxane derivatives have been prepared following a hydrolitic approach by hydrolisis-polycondensation of carboranylchlorosilane or carboranylethoxysilane. A second approach was a non hydrolytic route using carboranylchlorosilane and DMSO as oxygen source.
 
In parallel, we have also worked on the anchoring of cobaltabisdicarbollide derivatives on the surface of \ce{TiO2} nanoparticles and oxidized silicon wafers. Thus, adequate organophosphorous derivatives of \cosane\ have been prepared to be used as coupling molecules for the modification of titanium dioxide surfaces. The functionalization of the surface results from the formation of \ce{Ti-O-P} bridges by condensation of \ce{P-OH} groups with surface hydroxyl groups and coordination of the phosphoryl groups to surface Lewis acidic sites. Besides, for anchoring cobaltabisdicarbollide derivatives on the surface of an oxidized silicon wafer, two different approaches were used, both based on the ring-opening reaction of the 8--dioxanate \cdiox\ with amines or isocyanate functions previously anchored to the surfaces.

Thus, cobaltabisdicarbollide derivatives  have demostrated to be suitable groups for functionalization of dendrimers and other nanostructures  such as nanoparticles and wafers providing a large number of materials with interesting potential applications.





