%
% Paquetes que pueden serte de utilidad (rec = recomendado, opc = opcional)
%
\usepackage{fancyhdr}          % (rec)  permite cambiar varios par�metros de las cabeceras y pi�s de p�gina
%\usepackage{courier}           % (opc)  usa esta fuente por defecto
\usepackage{setspace}          % (opc)  permite cambiar el espaciado entre l�neas
\usepackage{longtable}         % (opc)  permite que las tablas ocupen varias p�ginas
\usepackage{lscape}            % (opc)  permite el uso del comando \landscape, para poner algo apaisado
\usepackage{color}             % (opc)  varios comandos relativos al color (como \color)
\usepackage{rotating}          % (opc)  permite rotar PSs y EPSs
\usepackage{textcomp}          % (opc)  permite incluir el s�mbolo del euro, con \texteuro
\usepackage[spanish]{minitoc}           % (opc)  permite incluir ToCs (�ndice de materias) para cada cap�tulo
\usepackage{epsf}              % (opc)  permite ciertas manipulaciones a EPSs
\usepackage[absolute]{textpos} % (rec)  permite posicionado arbitrario de texto (necesario para la portada)
\usepackage[spanish]{babel}   % (rec)  da soporte para castellano a LaTeX
\usepackage[latin1]{inputenc}  % (opc)  permite introducir caracteres como �, etc, en el input
\usepackage{latexsym} % S�mbolos 
\usepackage{graphicx} % Inclusi�n de gr�ficos. Soporte para \figura (m�s abajo) 
\usepackage{Config/bpchem}
\usepackage[version=3]{mhchem}
\usepackage[sort&compress,super,comma]{natbib}
\usepackage{notes2bib}
\usepackage{mciteplus}
\usepackage{xkeyval}
\usepackage{enumerate}
\sloppy % suaviza las reglas de ruptura de l�neas de LaTeX
\frenchspacing % usar espaciado normal despu�s de '.'
\pagestyle{headings} % p�ginas con encabezado y pie b�sico

%
% Settings para los m�rgenes. Descomenta y modifica si sabes lo que haces. N�tese
% que a los valores dados se les a�ade una pulgada extra. Los valores dados son los
% predeterminados para papel A4 y el estilo itsas_pfc.cls.
%
\setlength{\oddsidemargin}{0pt}     % m�rgen izquierdo para p�ginas impares (izquierda) default 40pt
\setlength{\evensidemargin}{0pt}    % m�rgen izquierdo para p�ginas pares (derecha) default 10pt
\setlength{\textwidth}{450pt}        % anchura del cuerpo de texto default 400pt

%
% Recomendado para mejorar la colocaci�n autom�tica de las figuras.
% (tomado de http://dcwww.camp.dtu.dk/~schiotz/comp/LatexTips/LatexTips.html#captfont)
%
\renewcommand{\topfraction}{0.85}
\renewcommand{\textfraction}{0.1}
\renewcommand{\floatpagefraction}{0.75}

\addto\captionsspanish{\renewcommand*{\tablename}{Tabla}}

%formateo de los numero y letras de las referencias cruzadas sacadas de una sublista
\renewcommand{\mcitesubrefform}{$^{\arabic{mcitebibitemcount}\alph{mcitesubitemcount}}$}
%\providecommand{\mcitesubrefform}{\arabic{mcitebibitemcount}.\alph{mcitesubitemcount}}

%quita ``indice general'' en el minitoc
\renewcommand{\mtctitle}{}

% Espacio entre el borde superior de la p�gina y donde comienza el texto (ah� van las
% cabeceras). LaTeX se queja si usamos el paquete fanchyhdr y headheight es menor de 15pt
%
\headheight 15pt

%
% Para el paquete textpos (usado para la portada)
%
\setlength{\TPHorizModule}{\paperwidth}
\setlength{\TPVertModule}{\paperheight}
\newcommand{\tb}[4]{\begin{textblock}{#1}[0.5,0.5](#2,#3)\begin{center}#4\end{center}\end{textblock}}

%
% Aqu� puedes definir tus comandos.
% 
% \newcommand{cmd}[args]{def}
%
% cmd  = el comando a definir (p.e. \cadena)
% args = el n�mero de argumentos
% def  = la definici�n, sustituyendo #1, #2... por el primer, segundo... argumento
%
% Por ejemplo:
%
% \newcommand{\agua}[1]{H\ensuremath{_#1}O}
%
% Cada vez que escribamos "\agua{33}", en el output saldr�: "H33O" (con el 33 como sub�ndice)
%

\newcommand{\Cscosane}{\BPChem{Cs[3,3'-Co(1,2-C\_2B\_9H\_{11})\_2]}}
\newcommand{\cosane}{\BPChem{[3,3'-Co(1,2-C\_2B\_9H\_{11})\_2]\^{--}}}
\newcommand{\cosaneSiH}{\BPChem{[1,1'-$\mu$-SiMeH-3,3'-Co(1,2-C\_2B\_9H\_{10})\_2]\^{--}}}
\newcommand{\cosanep}{\BPChem{[8,8'-$\mu$-C\_6H\_4-3,3'-Co(1,2-C\_2B\_9H\_{10})\_2]\^{--}}}
\newcommand{\Cscosanep}{\BPChem{Cs[8,8'-$\mu$-C\_6H\_4-3,3'-Co(1,2-C\_2B\_9H\_{10})\_2]}}
\newcommand{\cosanefrag}{\BPChem{[1,1'-$\mu$-Si(CH\_2-)(CH\_3)-3,3'-Co(1,2-C\_2B\_9H\_{10})\_2]\^{--}}}
\newcommand{\dendron}{\BPChem{[1,1'-$\mu$-Si(CH\_3)\{(CH\_2)\_2-Si(CHCH\_2)\_3\}-3,3'-Co(1,2-C\_2B\_9H\_{10})\_2]\^{--}}}
\newcommand{\cocho}{\BPChem{[1,1'-$\mu$-SiMe\_2-8,8'-$\mu$-C\_6H\_4-3,3'-Co(1,2-C\_2B\_9H\_{10})\_2]\^{--}}}
\newcommand{\ccuatro}{\BPChem{[1,1'-$\mu$-SiMe\_2-3,3'-Co(1,2-C\_2B\_9H\_{10})\_2]\^{--}}}
\newcommand{\ctres}{\BPChem{[1-SiMe\_2H-3,3'-Co(1,2-C\_2B\_9H\_{10})(1',2'-C\_2B\_9H\_{11})]\^{--}}}
\newcommand{\cdiox}{\BPChem{[3,3'-Co(8-C\_4H\_8O\_2-1,2-C\_2B\_9H\_{10})(1',2'-C\_2B\_9H\_{11})]}}
\newcommand{\cseis}{Cs\BPChem{[1-SiMe\_3-3,3'-Co(1,2-C\_2B\_9H\_{10})(1',2'-C\_2B\_9H\_{11})]}}
\newcommand{\csiete}{Cs\BPChem{[1,1'-(SiMe\_3)\_2-3,3'-Co(1,2-C\_2B\_9H\_{10})\_2]}}
\newcommand{\cnueve}{Cs\BPChem{[8,8'-$\mu$-(C\_6H\_4)-1,1'-$\mu$-SiMeH-3,3'-Co(1,2-C\_2B\_9H\_{9})\_2]}}
\newcommand{\cdiez}{Cs\BPChem{[8,8'-$\mu$-(C\_6H\_4)-1-SiMe\_3-3,3'-Co(1,2-C\_2B\_9H\_{9})(1',2'-C\_2B\_9H\_{10})]}}
\newcommand{\ccinco}{\BPChem{[1,1'-$\mu$-SiMeH-3,3'-Co(1,2-C\_2B\_9H\_{10})\_2]\^{--}}}


%pruebas entre bpchem y ce
%\ce{[3 , 3' - Co( 1 ,2 - C_2B_9H_{11})_2]^-}

%\BPChem{[3,3'-Co(1,2-C\_2B\_9H\_{11})\_2]\^-}

% Aqu� puedes instruir a LaTeX de por d�nde cortar las palabras que �l autom�ticamente
% no sepa. P.e., para cortar "gnomonly" solo por donde se se�ala con guiones (-).
%
\hyphenation{gno-mon-ly} 
 
%
% Que las primeras p�ginas sean numeradas con n�meros romanos.
% M�s adelante se cambiar� de nuevo a ar�bicos.
%
\pagenumbering{Roman}
%\usepackage{hyperref}
