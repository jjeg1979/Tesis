\tocchapter{Conclusiones}

\begin{enumerate}
\item The first \ce{C_c}-mono and \ce{C_c}-disubstituted cobaltabisdicarbollide derivatives containing different organosilane functions have been successfully prepared by the direct reaction of the mono or dilithium salts of starting anions, \ce{\textbf{[1]}^-} and \ce{\textbf{[2]}^-}, with the appropriate chlorosilanes and under careful control of the temperature. The reaction temperature was a key factor, because at very low temperatures (78 �C) \ce{C_c}-monosubstituted species and high isomeric purity were obtained, whereas increasing the temperature led to \ce{C_c}-disubstituted anions and structural isomers mixtures.

\item Density functional theory (DFT) at the B3LYP/6-311G (d,p) level was applied to optimise the geometries of the prepared silyl-containing cobaltabisdicarbollide derivatives, \ce{\textbf{[3]}^-}--\ce{\textbf{[10]}^-}, and calculate their relative energies. The theoretical studies perfectly agree with the experimental results, indicating that racemic mixtures (\textit{rac} isomers) are more stable than \textit{meso} isomers.

\item The anion \ctres, \ce{\textbf{[3]}^-}, represents the first example of a \ce{C_c}-monosubstituted cobaltabisdicarbollide fully characterised by X-ray diffraction. The crystal structure shows three H$\cdot \cdot \cdot$H short contacts: two \ce{Si-H}$\cdot \cdot \cdot$\ce{H-C_c} and one \ce{Si-CH2-H}$\cdot \cdot \cdot$\ce{H-C_c} contact. The shortest one corresponds to \ce{Si-CH2-H}$\cdot \cdot \cdot$\ce{H-C_c} with a H$\cdot \cdot \cdot$H distance of 2.059 \AA{}, whereas the two longest correspond to \ce{Si-H}$\cdot \cdot \cdot$\ce{H-C_c} with 2.212 and 2.409 \AA{}, respectively. However, by using QTAIM and Charge Analyses Population on the hydrogen atoms, it has been conclude that the \ce{C-H}$\cdot \cdot \cdot$\ce{H-C_c}, is not a dihydrogen bond (DHB) or it is a weak \ce{H-H} interaction. On the contrary, both \ce{Si-H}$\cdot \cdot \cdot$\ce{H-C_c} interactions are DHB and can be considered part of an asymmetric bifurcated DHB. 


\item Compounds \ccuatro, \ce{\textbf{[4]}^-} and \cocho, \ce{\textbf{[8]}^-}, that contain a bridge (-$\mu$\ce{- SiMe2 -}) between both dicarbollide ligands, were obtained unexpectedly from the reaction of the respective monolithium salt of \ce{\textbf{[1]}^-} and \ce{\textbf{[2]}^-} with \ce{Me2SiHCl} at low temperatures. A hypothetical mechanism has been proposed to explain the formation of these compounds through an intramolecular reaction, that implies the reaction of an acidic \ce{C_c-H} with the \ce{Si-H} hydride, and the loss of hydrogen. This has been supported by theoretical studies, that are related to the crystal structure of anion \ce{\textbf{[3]}^-}.



\item A trifunctional molecule containing a cobaltacarborane and three vinylsilane functions, \ce{\textbf{[11]}^-};  as well as, two families of polyanionic carbosilane and cyclic carbosiloxane metallodendrimers peripherally decorated with four or eight cobaltabisdicarbollide moieties, \ce{\textbf{[12]}^{4-}}--\ce{\textbf{[17]}^{8-}}, have been prepared by hydrosilylation of the suitable dendritic molecules containing terminal \ce{C=C} functionalities by using the anion \ccinco, \ce{\textbf{[5]}^{-}}, in the presence of Karstedt catalyst and optimized reaction conditions. The reaction were monitored by  $^1$H-NMR spectroscopy by the desappearance of vinyl-functions. 

\item Polyanionic boron-rich metallodendrimers based on poly (aryl-eter) dendrimers with the fluorescence triphenilbenzene (TFB) core, and allyl-terminated functions at the surface, have been functionalizated with \ce{\textbf{[5]}^{-}} to achieve the metallodendrimers \ce{\textbf{[18]}^{3-}}, \ce{\textbf{[19]}^{6-}} and \ce{\textbf{[20]}^{12-}} that contain three, six and twelve cobaltacarboranes at the periphery. To our knowledge, the last represents the high metallacarborane containing molecule describe in the literature.

\item Poly(alkyl aryl-ether)  type star-shape molecules and dendrimers were decorated by metallacarboranes by using the ring-opening reaction of 8--dioxanate \BPChem{[3,3'-Co(8-C\_4H\_8O\_2-1,2-C\_2B\_9H\_{10})(1',2'-C\_2B\_9H\_{11})]}, by the nucleophilic attack with the alcoholate functions, obtained by deprotonation of alcohol groups (\ce{- OH}) located at the starting dendrimers periphery, to give polyanionic species \ce{\textbf{[21]}^{4-}}--\ce{\textbf{[24]}^{8-}} with high-boron-content. 

\item All metallodendrimers have been characterized by FT-IR, $^1$H, $^{11}$B, $^{13}$C and $^{29}$Si NMR and UV-Vis spectroscopy, and in some cases elemental analysis and mass spectrometry (MALDI-TOF or ESI). However, for dendrimers with the highest molecular weights it was not possible to obtain the mass spectra, due to the great fragmentation.

\item The UV-Vis spectroscopy have shown a linear relationship between the absortivity and the numbers of cobaltabisdicarbollides located at the periphery. Thus, this technique  was used as an undirected method to corroborate the full functionalization of dendrimers with cobaltabisdicarbollide moieties, and subsequently confirm the unified character of the dendritic macromolecules. 

\item The UV-Vis spectroscopy has also been a good tool for the study of the carbosilane and cyclic carbosiloxane metallodendrimers solubility in water/DMSO solutions, by measuring the absorptivities of different metallodendrimer solutions. 

\item Carboranyl-containing disiloxanes, cyclic-siloxane and cage-like silsesquioxane have been prepared in high yields. Two routes are compared for their preparation: a classical hydrolytic process based on hydrolysis and condensation of the adequated carboranylchlorosilane and carboranylethoxysilane precursors and a non-hydrolytic route based on the specific reactivity of chorosilane toward DMSO, that is the oxygen source. Based on the typical reactivity of the carboranyl group toward nucleophiles, dianionic disiloxanes (\ce{\textbf{[41]}^{2-}} and \ce{\textbf{[42]}^{2-}}) and octaanionic silsesquioxanes (\ce{\textbf{[43]}^{8-}}) were obtained without modification of the siloxane bond.  The present results have pointed out the efficiency of the non-hydrolytic route with DMSO, that is particularly attractive for limiting the formation of linear oligopolysiloxane. Products are fully characterized by FTIR, NMR and MALDI-TOF methods.

\item Two phosphorus-containing cobaltabisdicarbollide derivatives, \ce{\textbf{[44]}^{-}} and \ce{\textbf{[45]}^{-}}, have been prepared to modify the surface of titanium dioxide particle, following an experimental procedure previously described. The \ce{TiO2} particles were reacted with a solution of the phosphate or phosphinate coupling molecules in a 5-fold excess relative to the amount needed for a full surface coverage on the particles, to obtain \textbf{44}@\ce{TiO2} and \textbf{45}@\ce{TiO2}, respectively. These surfaces are fully characterized by FTIR and $^{31}$P and $^{11}$B CP-MAS-NMR. 


\item Anchoring cobaltabisdicarbollide derivatives on the \ce{SiO2} surface of Si wafers has been achieved by using two approaches. The first approach is a ``in situ'' ring-opening reaction of 8--dioxanate \BPChem{[3,3'-Co(8-C\_4H\_8O\_2-1,2-C\_2B\_9H\_{10})(1',2'-C\_2B\_9H\_{11})]} by nucleophilic attack of previously anchorated amines in the \ce{SiO2} surface. The second approach is the reaction of the amine-terminated cobaltabisdicarbollide Cs[\textbf{47}] with an isocyanate group previously anchoraded in the \ce{SiO2} surface to give an urea connection. Both surfaces, \textbf{46}@\ce{SiO2} and Cs\textbf{47}@\ce{SiO2} have been characterized by FTIR-ATR, contact angle, AFM, UV-Vis, ellipsometry and XPS.

\end{enumerate}
