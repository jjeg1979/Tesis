\tocchapter{Objetivos}

La preparaci�n y caracterizaci�n de nuevos materiales tienen gran inter�s cient�fico debido al gran abanico de posibles aplicaciones. La creaci�n de nuevas v�as para la modificaci�n qu�mica de la periferia de dendr�meros, recubrimientos de superficies y funcionalizaci�n de nanopart�culas, abre la puerta a nuevas posibilidades de aplicaci�n o modificaci�n de las propiedades que presentan estos materiales.

En los �ltimos a�os, nuestro grupo se ha interesado en la preparaci�n de compuestos ricos en boro que muestran una cierta tendencia a ser solubles en agua y potenciales aplicaciones en medicina. El objetivo general de este trabajo es desarrollar estrategias para la incorporaci�n de derivados de carborano, y en particular del cobaltacarborano \cosane\ a diferentes plataformas con el fin de preparar, por un lado, compuestos ani�nicos ricos en boro, y por otro lado, estudiar las propiedades que estos cl�steres les puedan transmitir. Para ello se han establecido los siguientes objetivos concretos:


\begin{enumerate}
\item Sintetizar y caracterizar derivados del \cosane\ con funciones \ce{Si-H} apropiadas para hacer hidrosililaci�n sobre dobles enlaces en la periferia de un dendr�mero.
\item Sintetizar estructuras dendrim�ricas de tipo carbosilano y ciclocarbosiloxano de diferentes generaciones con grupos vinilo terminales, para despu�s funcionalizarlas mediante hidrosililaci�n con un derivado del \cosane\ que contenga un grupo \ce{Si-H}. 
\item Funcionalizar estructuras dendrim�ricas de tipo poli(aril-�ter) (dendr�meros tipo Fr�chet) de distinta generaci�n que presentan grupos alilo terminales, mediante reacciones de hidrosililaci�n con derivados del cobaltacarborano.
\item Funcionalizar estructuras dendrim�ricas de tipo poli(aril-�ter) que posean  grupos -OH en la periferia, mediante la apertura de anillo dioxano del derivado  de cobaltacarborano, \cdiox.
\item Obtenci�n de siloxanos, ciclosiloxanos y octasilsesquioxanos que contengan cl�steres de carborano, a partir de carboranilclorosilanos y carboraniletoxisilanos. 
\item Preparaci�n de derivados fosforados del \cosane\ adecuados para funcionalizar la superficie de nanopart�culas de \ce{TiO2}.
\item Preparaci�n de derivados de \cosane\ adecuados para ensamblar a superficies de \textit{wafers} de silicio oxidado. 
\end{enumerate}




