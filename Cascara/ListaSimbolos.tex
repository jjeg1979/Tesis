\chapter*{Lista de S�mbolos}
{\Large \textbf{Latinos}}\\
\hrule
\normalsize
~        \hfill   ~\\
$A$ \hfill �rea proyectada.\\
$a$ \hfill Coeficiente de correlaci�n.\\
$b$ \hfill Coeficiente de correlaci�n.\\
\rightline{Distancia.}\\
$C$ \hfill Brillo.\\
$c$ \hfill C�rculo.\\
\rightline{Coeficiente de correlaci�n.}\\
$D$ \hfill Dimensi�n.\\
$d$ \hfill Di�metro.\\
\rightline{Distancia.}\\
$h$ \hfill Factor incluido en la expresi�n para simplificar la expresi�n de la funci�n $\mu$.\\
$i$	\hfill Numeral de la part�cula primaria en el aglomerado.\\
$I$ \hfill Intensidad de radiaci�n.\\
\rightline{Momento de inercia.}\\
$J$	\hfill N�mero de coordinaci�n.\\
$K$	\hfill Coeficiente.\\
$k$	\hfill Prefactor.\\
$L$	\hfill Espesor.\\
$m$	\hfill Masa.\\
\rightline{Par�metro de forma de las funciones matem�ticas.}\\
$N$	\hfill N�mero de estructuras elementales.\\
$n$	\hfill N�mero.\\
$p$	\hfill Funci�n de empaquetamiento.\\
$r$	\hfill Radio.\\
\rightline{Distancia desde el centro de gravedad.}\\
$V$	\hfill Volumen.\\
$X$	\hfill Coordenada.\\
$Y$	\hfill Coordenada.\\
$Z$	\hfill Coordenada.\\
$z$	\hfill Altura.\\

\newpage

{\Large \textbf{S�mbolos griegos}}\\
\hrule
\normalsize
~        \hfill   ~\\
$\alpha$ \hfill �ngulo de aplastamiento.\\
\rightline{Funci�n que modifica la masa de una part�cula aplastada.}\\
$\beta$ \hfill Funci�n que modifica el momento de inercia de una part�cula aplastada.\\
$\delta$ \hfill Coeficiente de aplastamiento.\\
$\gamma$ \hfill Funci�n que modifica el centro de gravedad de una part�cula (multi)aplastada.\\
$\rho$ \hfill Densidad.\\
$\mu$ \hfill Funci�n que relaciona $n_{p_o}$ y $i_n$ para los casos en los cuales $D_f\,=\,2, N\,=\,1$.\\
$\nu$ \hfill Funci�n que relaciona $n_{p_o}$ y $i_n$ para los casos en los cuales $D_f\,=\,2, N\,=\,2$.\\
$\eta$ \hfill Funci�n que relaciona $n_{p_o}$ y $i_n$ para los casos en los cuales $D_f\,=\,2, N\,=\,3$.\\
$\varphi$ \hfill �ngulo azimutal.\\
$\theta$ \hfill �ngulo polar.\\

{\Large \textbf{Sub�ndices}}\\
\hrule
\normalsize
~        \hfill   ~\\
$bcc$ \hfill C�bico centrado en el cuerpo.\\
$c$ \hfill Cono.\\
$ext$ \hfill Extinci�n de luz.\\
$f$ \hfill Fractal.\\
$G$ \hfill Centro de gravedad.\\
$g$ \hfill Giro.\\
$hc$ \hfill Empaquetamiento hexagonal compacto.\\
$mss$ \hfill Esfera multiaplastada.\\
$n$ \hfill N�meral.\\
$o$ \hfill Centro geom�trico.\\
$p$ \hfill Part�cula.\\
$pix$ \hfill P�xel.\\
$p_{o}$ \hfill Part�cula primaria.\\
$s$ \hfill Holl�n.\\
\rightline{Aplastamiento.}\\
$sc$ \hfill C�bico simple.\\
\rightline{Casquete esf�rico.}\\
$sss$ \hfill Esfera simplemente aplastada.\\