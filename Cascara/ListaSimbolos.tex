\chapter*{Lista de Símbolos}
{\Large \textbf{Latinos}}\\
\hrule
\normalsize
~        \hfill   ~\\
$A$ \hfill Área proyectada.\\
$a$ \hfill Coeficiente de correlación.\\
$b$ \hfill Coeficiente de correlación.\\
\rightline{Distancia.}\\
$C$ \hfill Brillo.\\
$c$ \hfill Círculo.\\
\rightline{Coeficiente de correlación.}\\
$D$ \hfill Dimensión.\\
$d$ \hfill Diámetro.\\
\rightline{Distancia.}\\
$h$ \hfill Factor incluido en la expresión para simplificar la expresión de la función $\mu$.\\
$i$	\hfill Numeral de la partícula primaria en el aglomerado.\\
$I$ \hfill Intensidad de radiación.\\
\rightline{Momento de inercia.}\\
$J$	\hfill Número de coordinación.\\
$K$	\hfill Coeficiente.\\
$k$	\hfill Prefactor.\\
$L$	\hfill Espesor.\\
$m$	\hfill Masa.\\
\rightline{Parámetro de forma de las funciones matemáticas.}\\
$N$	\hfill Número de estructuras elementales.\\
$n$	\hfill Número.\\
$p$	\hfill Función de empaquetamiento.\\
$r$	\hfill Radio.\\
\rightline{Distancia desde el centro de gravedad.}\\
$V$	\hfill Volumen.\\
$X$	\hfill Coordenada.\\
$Y$	\hfill Coordenada.\\
$Z$	\hfill Coordenada.\\
$z$	\hfill Altura.\\

\newpage

{\Large \textbf{Símbolos griegos}}\\
\hrule
\normalsize
~        \hfill   ~\\
$\alpha$ \hfill Ángulo de aplastamiento.\\
\rightline{Función que modifica la masa de una partícula aplastada.}\\
$\beta$ \hfill Función que modifica el momento de inercia de una partícula aplastada.\\
$\delta$ \hfill Coeficiente de aplastamiento.\\
$\gamma$ \hfill Función que modifica el centro de gravedad de una partícula (multi)aplastada.\\
$\rho$ \hfill Densidad.\\
$\mu$ \hfill Función que relaciona $n_{p_o}$ y $i_n$ para los casos en los cuales $D_f\,=\,2, N\,=\,1$.\\
$\nu$ \hfill Función que relaciona $n_{p_o}$ y $i_n$ para los casos en los cuales $D_f\,=\,2, N\,=\,2$.\\
$\eta$ \hfill Función que relaciona $n_{p_o}$ y $i_n$ para los casos en los cuales $D_f\,=\,2, N\,=\,3$.\\
$\varphi$ \hfill Ángulo azimutal.\\
$\theta$ \hfill Ángulo polar.\\

{\Large \textbf{Subíndices}}\\
\hrule
\normalsize
~        \hfill   ~\\
$bcc$ \hfill Cúbico centrado en el cuerpo.\\
$c$ \hfill Cono.\\
$ext$ \hfill Extinción de luz.\\
$f$ \hfill Fractal.\\
$G$ \hfill Centro de gravedad.\\
$g$ \hfill Giro.\\
$hc$ \hfill Empaquetamiento hexagonal compacto.\\
$mss$ \hfill Esfera multiaplastada.\\
$n$ \hfill Númeral.\\
$o$ \hfill Centro geométrico.\\
$p$ \hfill Partícula.\\
$pix$ \hfill Píxel.\\
$p_{o}$ \hfill Partícula primaria.\\
$s$ \hfill Hollín.\\
\rightline{Aplastamiento.}\\
$sc$ \hfill Cúbico simple.\\
\rightline{Casquete esférico.}\\
$sss$ \hfill Esfera simplemente aplastada.\\