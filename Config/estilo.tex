% ====================================================== STYLE DOCUMENT ========================================================================
% This file contains all the neccesary package calls for the Thesis
% ===============================================================================================================================================
% inputenc for iso-8859-1 encoding
% fontenc for the font encoding
\usepackage[latin1]{inputenc}
\usepackage[T1]{fontenc}
\usepackage{mathpazo}		% Defines Mathpatho font
\linespread{1.05}         	% Palatino needs more leading (space between lines)

% \usepackage{natbib}
% sectionbib allows for as per chapter bibliography. It must be added before babel package
% sectionbib: For chapter bibliographies
% duplicate: To have a bibliography-by-chapter at the end
\usepackage[rootbib]{chapterbib}		% For the first Latex Compilation and BibTex Compilation for the master document
%\usepackage{chapterbib}
% \usepackage[sectionbib,duplicate]{chapterbib}
% \CitationPrefix{\thechapter.}				% Applies the chapter number as a prefix for citations
% This package also includes bibliography per chapter
% globalcitecopy: 
% sectionbib:
%\usepackage[globalcitecopy,sectionbib]{bibunits}

% Trial with biblatex to get good bibliography per chapter
% \usepackage[style=alphabetic]{biblatex}
% \addbibresource{References/Tesis.bib}

% Bable defines the current languages used
\usepackage[main=spanish,english]{babel}

% cite package allows as to manipulate how cites are formed. This package is not compatible with biblatex
\usepackage{cite}

% Use aditional symbols (AMS)
\usepackage{amsmath,amssymb,amsfonts,latexsym}

% Packet geometry to design document
\usepackage[left=2.5cm,right=2.5cm,top=2.5cm,bottom=2.5cm,footskip=1cm]{geometry}
\parindent = 5mm		% Sangr�a

% fixltx2e allows for the use of subscripts and superscripts inside the text
\usepackage{fixltx2e}

% enumitem package allows the changing of enumeration style (roman numbers...)
\usepackage{enumitem}

% Solve some issues with babel package in book
\renewcommand{\contentsname}{Contenido}
\renewcommand{\partname}{Parte}
\renewcommand{\appendixname}{Ap�ndice}
\renewcommand{\figurename}{Figura}
\renewcommand{\listtablename}{�ndice de Tablas}
\renewcommand\listfigurename{�ndice de Figuras}
\renewcommand{\chaptername}{Cap�tulo}
\renewcommand{\bibname}{Referencias}
\renewcommand\citeform[1]{(#1)}			% Changes reference format to [(1),(2),(3)]
\usepackage[nottoc]{tocbibind}
% backref package allows for backreferencing citations in the bibliography when used in per chapter bibliography
% ref: section number for reference. This package is not compatible with biblatex
\usepackage[ref]{backref}
% Graphicx package for using graphics
\usepackage{graphicx}
% Multirow package allows for the creation of tables
\usepackage{multirow}
% Titleref provides a command \titleref to cross-reference the titles of sections
\usepackage{titleref}
% subfig package allows for sub-floats (figures, tables) to be separately captioned, referenced and included in a list-of-floats page
\usepackage{subfig}
% minitoc creates a mini-table of contents at the beginning of each section. In the main text you have to add \dominitoc
%\usepackage{minitoc}
% titletoc is the same as minitoc
\usepackage{titletoc}
% makeidx creates an index
\usepackage{makeidx}
% nomencl creates a nomenclature
% intoc: Puts the nomenclature in the TOC
% spanish: For Spanish Language
\usepackage[intoc,spanish]{nomencl}
% glossaries is more up to date than nomencl
%\usepackage[xindy,nonumberlist,toc,nopostdot,style=altlist,nogroupskip]{glossaries}
\renewcommand{\nomname}{Lista de S�mbolos}


% color: Package to use color
% usenames, dvipsnames and svgnames give access to a lot of colors
% table is for including color in tables
\usepackage[usenames,dvipsnames,svgnames,table]{xcolor}

% Define maximum depth for minitoc
%\setcounter{minitocdepth}{3} 
% Rename Contents in minitoc
%\renewcommand{\mtctitle}{Contenido del Cap�tulo}

% hyperref is used for hyperlinks
% colorlinks: To show links with color
% linkcolor = color: Color of links
% urlcolor = color: Color of url links
% citecolor = color: Color of bibliographic references
% linktoc = all: For numbers in TOC colored as well
% backref = For backreferencing the citations in the global bibliography
\usepackage[colorlinks=true,linkcolor=red,urlcolor=blue,citecolor=blue,linktoc=all,backref=section]{hyperref}

% The following commands allow the subsubsubsection to be numbered in the text and in the index
\setcounter{secnumdepth}{4}		% For the text
\setcounter{tocdepth}{4}		% For the index
% This package allows the Reference to appear in the TOC without numbering
\usepackage[nottoc]{tocbibind}

% Rename Command in backref so the sections are in brackets
\renewcommand*{\backref}[1]{[#1]}
% Define Spanish words for backreferencing
\def\backrefspanish{%
	\def\backrefpagesname{p�ginas}%
	\def\backrefsectionsname{secciones}%
	\def\backrefsep{, }
	\def\backreftwosep{ y~}%
	\def\backreflastsep{ y~}%
}

% lettrine used capital letter
\usepackage{lettrine}

% Define the name of Tables
\usepackage[format=plain,justification=centerlast,width=15cm,labelsep=period,tablename=Tabla,skip=5pt]{caption}

% Define directories where to find figures
\graphicspath{{Figuras/}{Chapter1/Chapter1Figs/}{Chapter2/Chapter2Figs/}{Chapter3/Chapter3Figs/}}

% To customize headers and footers this package is a must. 
\usepackage{fancyhdr}

% lastpage package allows for the use of the lastpage
\usepackage{lastpage}

% appendix allows for more control over the appendixes format
\usepackage{appendix}

% -----------------------------------------------------------------------------------------------------------------------------------------------
% CODE ZONE
% -----------------------------------------------------------------------------------------------------------------------------------------------
% This code forbids the inserted pages to have either header or footer
%\makeatletter
%  \def\cleardoublepage{\clearpage\if@twoside \ifodd\c@page\else
%  \vspace*{\fill}
%    \thispagestyle{empty}
%    \newpage
%    \if@twocolumn\hbox{}\newpage\fi\fi\fi}
%\makeatother
