%=====================================PACKAGES=====================================
% Allows to change several parameters from headers and footers
\usepackage{fancyhdr}

% Allows to change space between lines
\usepackage{setspace}          

% Allows tables to occupy several pages
\usepackage{longtable}         

% Landscape command is in the following package
\usepackage{lscape} 

% Packet geometry to design document
\usepackage[left=2.5cm,right=2.5cm,top=2.5cm,bottom=2.5cm,footskip=1cm]{geometry}
\parindent = 5mm		% Sangría

% fixltx2e allows for the use of subscripts and superscripts inside the text
\usepackage{fixltx2e}           

% Commands relative to color
% color: Package to use color
% usenames, dvipsnames and svgnames give access to a lot of colors
% table is for including color in tables
\usepackage[usenames,dvipsnames,svgnames,table]{xcolor}

% Package to change color to references (taken from: http://tex.stackexchange.com/questions/197293/change-the-color-of-cite-number-in-bibliography)
\usepackage{xpatch}

% Allows to rotate PSs and EPSs
\usepackage{rotating}          

% Euro symbol
\usepackage{textcomp}          %

% Allows to include several TOC in every chapter
%\usepackage[spanish]{minitoc}           
\usepackage{titletoc}

% Manipulations for EPSs
\usepackage{epsf}            

% This is useful for the frontpage (arbitrary text positioning)
\usepackage[absolute]{textpos} 

% Spanish support and encondig (two following packages)
\usepackage[main=spanish, english]{babel}   
\usepackage[utf8]{inputenc} 

% Standard package for selecting font encodings
\usepackage[T1]{fontenc}

% Fonts to typeset mathematics to match Palatino
\usepackage{mathpazo}		% Defines Mathpatho font
\linespread{1.05}         	% Palatino needs more leading (space between lines)

% makeidx creates an index
\usepackage{makeidx}

% nomencl creates a nomenclature
% intoc: Puts the nomenclature in the TOC
% spanish: For Spanish Language
\usepackage[intoc,spanish]{nomencl}
% glossaries is more up to date than nomencl
%\usepackage[xindy,nonumberlist,toc,nopostdot,style=altlist,nogroupskip]{glossaries}

% Graphics inclusion, support for \figure (see below)
\usepackage{graphicx} 

% subfig package allows for sub-floats (figures, tables) to be separately captioned, referenced and included in a list-of-floats page
\usepackage{subfig}

% For the bibliography (natbib allows for multiple bibliographies in one document)
\usepackage[sort&compress,super,comma, square, sectionbib]{natbib}
\usepackage{chapterbib}

% Integrating notes into the bibliography
\usepackage{notes2bib}

% Enhanced multiple citations
\usepackage{mciteplus}

% cite package allows as to manipulate how cites are formed. This package is not compatible with biblatex
\usepackage{cite}

% Use aditional symbols (AMS)
\usepackage{amsmath,amssymb,amsfonts,latexsym}

% Extension of the keyval package
\usepackage{xkeyval}

% Enumerate with redefinable labels
\usepackage{enumerate}

% lettrine used capital letter
\usepackage{lettrine}

% Define the name of Tables
\usepackage[format=plain,justification=centerlast,width=15cm,labelsep=period,tablename=Tabla,skip=5pt]{caption}

% appendix allows for more control over the appendixes format
\usepackage{appendix}

% Add bibliography/index/contents to Table of Contents
\usepackage[nottoc]{tocbibind}

% hyperref is used for hyperlinks
% colorlinks: To show links with color
% linkcolor = color: Color of links
% urlcolor = color: Color of url links
% citecolor = color: Color of bibliographic references
% linktoc = all: For numbers in TOC colored as well
% backref = For backreferencing the citations in the global bibliography
\usepackage[colorlinks=true,linkcolor=red,urlcolor=blue,citecolor=blue,linktoc=all,backref=section]{hyperref}

% lastpage package allows for the use of the lastpage
\usepackage{lastpage}

% Titleref provides a command \titleref to cross-reference the titles of sections
\usepackage{titleref}

% backref package allows for backreferencing citations in the bibliography when used in per chapter bibliography
% ref: section number for reference. This package is not compatible with biblatex
%\usepackage[ref]{backref}

% Multirow package allows for the creation of tables
\usepackage{multirow}

% enumitem package allows the changing of enumeration style (roman numbers...)
\usepackage{enumitem}

% Define the name of Tables
\usepackage[format=plain,justification=centerlast,width=15cm,labelsep=period,tablename=Tabla,skip=5pt]{caption}

%==================================================================================

% Softens the line breaking rules from LaTeX
\sloppy 

% Changes decimal symbol to a point
\spanishdecimal{.}		

% To use normal spacing after '.'
\frenchspacing 

% The following commands allow the subsubsubsection to be numbered in the text and in the index
\setcounter{secnumdepth}{4}		% For the text
\setcounter{tocdepth}{4}		% For the index


% Basic header and footer
\pagestyle{headings} 

% Margin settings
%\setlength{\oddsidemargin}{0pt}     % Left margin for odd pages default 40pt
%\setlength{\evensidemargin}{0pt}    % Left margin even pages default 10pt
%\setlength{\textwidth}{450pt}       % Width of the body (default 400 pt)

%
% Recomendation to enhance the figures location
% (taken from http://dcwww.camp.dtu.dk/~schiotz/comp/LatexTips/LatexTips.html#captfont)
%
\renewcommand{\topfraction}{0.85}
\renewcommand{\textfraction}{0.1}
\renewcommand{\floatpagefraction}{0.75}

% To avoid Unicode char \u9: not set up for use with LaTeX error (taken from: http://tex.stackexchange.com/questions/83440/inputenc-error-unicode-char-u8-not-set-up-for-use-with-latex)
\DeclareUnicodeCharacter{00A0}{ }

% Rename Tables to Tablas in Spanish
\addto\captionsspanish{\renewcommand*{\tablename}{Tabla}}

% Formating for letters and numbers for crossreferences taken from a sublist
\renewcommand{\mcitesubrefform}{$^{\arabic{mcitebibitemcount}\alph{mcitesubitemcount}}$} 
%\providecommand{\mcitesubrefform}{\arabic{mcitebibitemcount}.\alph{mcitesubitemcount}}

% Remove "General Index" from the minitoc
%\renewcommand{\mtctitle}{}

% Spacing from the upper border and where the main text body begins. LaTeX  complains if we use fanchyhdr and headheight is less than 15pt
%
%\headheight 28pt

%
% For textpos package (used for frontpage)
%
\setlength{\TPHorizModule}{\paperwidth}
\setlength{\TPVertModule}{\paperheight}
\newcommand{\tb}[4]{\begin{textblock}{#1}[0.5,0.5](#2,#3)\begin{center}#4\end{center}\end{textblock}}

%
% Command redefinition.
% 
% \newcommand{cmd}[args]{def}
%
% cmd  = command to be redefined (i.e. \char)
% args = number of arguments
% def  = definition, substituting #1, #2... por el first, second... argument
%
% For example:
%
% \newcommand{\water}[1]{H\ensuremath{_#1}O}
%
% Anytime we write "\water{33}", as output would yield: "H33O" (33 as subscript)
%

% Rename Nomenclature
\renewcommand{\nomname}{Lista de Símbolos}

% To change color to cite references in the References Section (not in the main text taken from: http://tex.stackexchange.com/questions/197293/change-the-color-of-cite-number-in-bibliography)
\makeatletter
\xpatchcmd{\@lbibitem}
{\item[\hfil}
{\item[\hfil\color{blue}}
{}{}
\makeatother

% Rename Command in backref so the sections are in brackets
\renewcommand*{\backref}[1]{[#1]}
% Define Spanish words for backreferencing
\def\backrefspanish{%
	\def\backrefpagesname{páginas}%
	\def\backrefsectionsname{secciones}%
	\def\backrefsep{, }
	\def\backreftwosep{ y~}%
	\def\backreflastsep{ y~}%
}

% First pages with Roman numbers
% Later they will be changed to Arabic
%
\pagenumbering{Roman}

% To print nomenclature
\makeglossary
\makenomenclature

% To print index
\makeindex

% The following commands allow the subsubsubsection to be numbered in the text and in the index
\setcounter{secnumdepth}{4}		% For the text
\setcounter{tocdepth}{4}		% For the index

% -----------------------------------------------------------------------------------------------------------------------------------------------
% CODE ZONE
% -----------------------------------------------------------------------------------------------------------------------------------------------
% This code forbids the inserted pages to have either header or footer
\makeatletter
  \def\cleardoublepage{\clearpage\if@twoside \ifodd\c@page\else
  \vspace*{\fill}
    \thispagestyle{empty}
    \newpage
    \if@twocolumn\hbox{}\newpage\fi\fi\fi}
\makeatother