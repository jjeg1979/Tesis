\pagestyle{fancy}                                % elegir este estilo de cpps (recomendado)
\fancyhf{}                                       % borra el estilo anterior para cpps, para luego redefinirlos
\fancyhead[LE,RO]{\textbf{\thepage}}             % Cabecera: n�mero de p�gina en negrita.

\fancyhead[RE]{\nouppercase{\leftmark}}          % Cabecera: incluye informaci�n del nivel superior (Cap�tulo)
                                                 % a la derecha (R) de las p�ginas pares (E), evitando escribir
						 % todo en may�sculas (que ser�a la opci�n por defecto).

\fancyhead[LO]{\nouppercase{\rightmark}}         % Cabecera: incluyer informaci�n del nivel inferior (Secci�n)
                                                 % a la izquierda (L) de las p�ginas impares (O), evitando escribir
						 % todo en may�sculas (que ser�a la opci�n por defecto).

\renewcommand{\headrulewidth}{0.5pt}             % Cabecera: subraya la cabecera (fijar en "0pt" si no se desea).
\renewcommand{\footrulewidth}{0pt}               % Pi�: subraya el pie de p�gina (fijar en "0pt" si no se desea).
